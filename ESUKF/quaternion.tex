% !TEX root=main.tex

\section{Quaternions}

A quaternion is a generalization of the complex numbers $a + bi \in \mathbb{C}$. We can think of quaternions as defining a new set $\mathbb{H} = \mathbb{C} + \mathbb{C}j$, and defining $k \triangleq ij$ we create can write any quaternion as
\beq
a + bi + cj + dk \in \mathbb{H}, 
\eeq
where $\{i,j,k\}$ are three imaginary units that multiply as follows:
\beq
i^2 = j^2 = k^2 = ijk = -1.
\eeq

Quaternions are powerful because they share many properties with complex numbers. Regarding quaternions as complex numbers with multidimensional imaginary part, we can rewrite every quaternion,
\beq
q_w + q_xi + q_yj + q_zk = 
\bma
1 & i & j & k
\ema
\bma
q_w \\
q_x \\
q_y \\
q_z \\
\ema
\eeq
We typically refer to this last vector as the quaternion and split it into its real scalar and imaginary vector components
\beq
\mathbf{q} = 
\bma
q_w \\
\mathbf{q_v} \\
\ema
\eeq
Continuing with this notation, we can now create an algebra for quaternions as follows: Let $\mathbf{q'},\mathbf{q} \in \mathbb{H}$ and $\alpha \in \mathbb{R}$.
\beq
\alpha\mathbf{q} = 
\bma
\alpha q_w \\
\alpha \mathbf{q_v}
\ema
\eeq
\beq
\mathbf{q'} \pm \mathbf{q} = 
\bma
q'_w \pm q_w \\
\mathbf{q'_v} \pm \mathbf{q_v} \\
\ema
\eeq

\beq
\mathbf{q'} \otimes \mathbf{q} =
\bma
q'_wq_w - \mathbf{q'}_v^\top\mathbf{q}_v \\
q'_w\mathbf{q_v} + q_w\mathbf{q'_v} + \mathbf{q'_v} \times \mathbf{q_v} \\
\ema
\eeq
At this point, we note that 
\beq
\mathbf{q'} \otimes \mathbf{q} \neq \mathbf{q} \otimes \mathbf{q'}
\eeq
Furthermore, we notice that the quaternion project distributes over the sum:
\beq
\mathbf{q'} \otimes (\mathbf{q} + \mathbf{q''}) = \mathbf{q'} \otimes \mathbf{q} + \mathbf{q'} \otimes \mathbf{q''}
\eeq
\beq
(\mathbf{q'} + \mathbf{q}) \otimes \mathbf{q''} = \mathbf{q'} \otimes \mathbf{q''} + \mathbf{q} \otimes \mathbf{q''} \\
\eeq

To make this algebra useful, we will also specify an identity element, a conjugate operator and an inverse for every $\mathbf{q} \in \mathbb{H}$. These are defined analgously to the complex numbers. The identity element is given by
\beq
\mathbf{1} = 
\bma
1 \\
\mathbf{0}
\ema
\eeq
We define the conjugate operation as follows
\beq
\mathbf{q}^* =
\bma
q_w \\
\mathbf{q_v}
\ema^*
=
\bma
q_w \\
-\mathbf{q_v}
\ema
\eeq
This conjugate operation shares some properties with the complex numbers such as
\beq
(\mathbf{q'} \otimes \mathbf{q})^* = \mathbf{q}^* \otimes \mathbf{q'}^*
\eeq
and
\beq
\mathbf{q} \otimes \mathbf{q}^* = \mathbf{q}^* \otimes \mathbf{q} = 
\bma
q_w^2 + ||\mathbf{q_v}||^2 \\
\mathbf{0}
\ema.
\eeq
This last property is used to define the norm of a quaternion
\beq
||\mathbf{q}|| = \sqrt{\mathbf{q} \otimes \mathbf{q}^*}.
\eeq
Now we can derive an inverse as follows
\beq
\begin{aligned}
  ||\mathbf{q}||^2 = \mathbf{q} \otimes \mathbf{q}^*  = \mathbf{q}^* \otimes \mathbf{q} \\
  \mathbf{1} = \mathbf{q} \otimes \frac{\mathbf{q}^*}{||\mathbf{q}||^2} =
  \frac{\mathbf{q}^*}{||\mathbf{q}||^2} \otimes \mathbf{q}.
\end{aligned}
\eeq
Thus we necessarily have that
\beq
\mathbf{q}^{-1} = \frac{\mathbf{q}^*}{||\mathbf{q}||^2}
\eeq
Note: in the special case that $||\mathbf{q}|| = 1$ we have that
\beq
\mathbf{q}^{-1} = \mathbf{q}^*.
\eeq
We call these quaternions ``unit quaternions''.


We will also specify an exponential and logarithmic mapping for quaternions. These are derived in Ref. \cite{Sola2017}. The derivation follows that of complex numbers and thus uses taylor series to decompose the exponential into sines and cosines. Specifically, we have that
\beq
e^{\mathbf{q}} = 
e^{q_w}\bma
\cos ||\mathbf{q_v}|| \\
\frac{\mathbf{q_v}}{||\mathbf{q_v}||} \sin ||\mathbf{q_v}||
\ema
\eeq
We note that when $\mathbf{q}$ is a pure quaternion, we have that $||e^{\mathbf{q}}|| = 1$. This implies that any unit quaternion can be parameterized as 
\beq
e^{
\bma
0 \\
\theta \mathbf{u}
\ema}
=
\bma
\cos \theta \\
\mathbf{u} \sin \theta
\ema
\eeq

\beq
\log \mathbf{q} = 
\bma
\log ||\mathbf{q}|| \\
\frac{\mathbf{q_v}}{||\mathbf{q_v}||} \arctan (||\mathbf{q_v}||, q_w).
\ema
\eeq
In the special case that $||q|| = 1$, the logarithm returns a ``pure'' quaternion, which is defined as a quaternion with $0$ real part.
